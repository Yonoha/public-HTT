\RequirePackage{plautopatch}
\documentclass[uplatex, a4paper, 14Q, dvipdfmx]{jsreport}
\usepackage{docmute}
\usepackage{../../mypackage}

\setcounter{secnumdepth}{4}
\title{1.2.7 高次圏の関手}
\author{Keima Akasaka}
\date{\today}

\begin{document}

% \maketitle

\setcounter{chapter}{1}
\setcounter{section}{2} 
\setcounter{subsection}{6}   
\setcounter{subsubsection}{1}

\subsection{高次圏の関手}

高次圏$\C$におけるホモトピー連接図式は関手$\J \to \C$の特別な場合である. 
通常の圏の集まりが関手を1射, 自然変換を2射とするような双圏をなすように, $\infty$圏の集まりが$\infty$双圏をなすようにしたい. 
このために, 任意の$\infty$圏$\C,\C'$に対して, $\C$から$\C'$への関手のなす$\infty$圏$\Fun(\C,\C')$を定義する必要がある. 

位相的圏の枠組みにおいて, 適切な$\Fun(\C,\C')$を定義することはとても難しい. 
$\Fun(\C,\C')$は$\C$から$\C'$への位相的関手のなす圏としたい.
つまり, 射空間の間の連続写像を定めるような関手を考える必要があるが, 1.2.6節で見たように, この対応は高次圏論では厳格すぎる. 

\begin{remark} \label{rem.1.2.7.1}
  モデル圏の言葉を使うと, この問題は任意の位相的圏がコファイブラントであるとは限らないことが原因である.
  位相的圏$\C$がコファイブラントのとき(例えば, $\C$がある単体的集合$S$を用いて$|\mathfrak{C}[S]|$として表されるとき), $\C$から$\C'$への位相的関手の集まりは$\C$から$\C'$への$\infty$圏的な関手をすべて含む程度には大きい.
  しかし, 通常の圏を位相的圏とみなすと, 一般にはコファイブラントではない.
  最も重要なことはコファイブラント性が直積で保たれないことである. 
  つまり, コファイブラント位相的圏$\C$に対して, $\C \times [1]$はコファイブラントではない. (これは$\C$が$[1]$の場合でもコファイブラントでない.)
  よって, $\C$がコファイブラントであったとしても, $\Fun(\C,\C')$の射空間をうまく定義することができない.
  これが高次圏のモデルとして位相的(もしくは単体的)圏を使うことの最も重要な技術的な欠点である. 
\end{remark}

$\infty$圏の枠組みにおける関手圏の構成は非常に簡単である. 
$\C,\D$を$\infty$圏とするとき, $\C$から$\D$への関手は単に単体的集合の射$\C \to \D$とすればよい. 

\begin{notation}[関手$\infty$圏] \label{nota.1.2.7.2}
  $\C,\D$を単体的集合とする. 
  任意の$n \geq 0$に対して, 
  \begin{align*}
    \Fun(\C,\D)_n := \Hom_{\sSet}(\C \times \Delta^n, \D)
  \end{align*}
  として, 単体的集合$\Fun(\C,\D)$を定義する. 
  この定義を$\D$が$\infty$圏の場合のみに用いて, $\Fun(\C,\D)$を$\C$から$\D$への関手$\infty$圏($\infty$-category of functors)という. 
  $\Fun(\C,\D)$における射を関手の自然変換(natural transformation)といい, $\Fun(\C,\D)$における同値を自然同値(natural equivalence)という. 
\end{notation}

次の命題は2.2.5節で証明する.

\begin{proposition} \label{prop.1.2.7.3}
  $K$を単体的集合とする. 
  このとき, 次が成立する. 
  \begin{enumerate}
    \item 任意の$\infty$圏$\C$に対して, 単体的集合$\Fun(K,\C)$は$\infty$圏である.
    \item $\C \to \D$を$\infty$圏の圏同値とする. 
    このとき, 誘導される射$\Fun(K,\C) \to \Fun(K,\D)$は圏同値である. 
    \item $\C$を$\infty$圏, $K \to K'$を単体的集合の圏同値とする. 
    このとき, 誘導される射$\Fun(K',\C) \to \Fun(K,\C)$は圏同値である. 
  \end{enumerate}
\end{proposition}

\end{document}