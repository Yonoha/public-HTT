\RequirePackage{plautopatch}
\documentclass[uplatex, a4paper, 14Q, dvipdfmx]{jsreport}
\usepackage{docmute}
\usepackage{../../mypackage}

\setcounter{secnumdepth}{4}
\title{1.2.11 \texorpdfstring{$\infty$}{infty}圏の部分圏}
\author{Keima Akasaka}
\date{\today}

\begin{document}

% \maketitle

\setcounter{chapter}{1}
\setcounter{section}{2} 
\setcounter{subsection}{10}   
\setcounter{subsubsection}{1}

\subsection{\texorpdfstring{$\infty$}{infty}圏の部分圏}

% https://math.stackexchange.com/questions/1489079/definition-of-the-notion-of-a-subcategory-of-an-infty-category?rq=1
通常の圏論と同様に, 高次圏における部分圏を定義する. 
$\C$を$\infty$圏, $\h\C$を$\C$のホモトピー圏, $(\h\C)'$を$\h\C$の部分圏とする. 
このとき, 次のプルバックで定義される$\C'$を$(\h\C)'$で貼られる$\C$の部分圏(subcategory)という. 
% https://q.uiver.app/#q=WzAsNCxbMCwwLCJcXEMnIl0sWzEsMCwiXFxDIl0sWzAsMSwiXFxOKChcXGhcXEMpJykiXSxbMSwxLCJcXE4oXFxoXFxDKSJdLFswLDFdLFswLDIsIiIsMix7InN0eWxlIjp7ImJvZHkiOnsibmFtZSI6ImRhc2hlZCJ9fX1dLFsxLDNdLFsyLDNdLFswLDMsIiIsMSx7InN0eWxlIjp7Im5hbWUiOiJjb3JuZXIifX1dXQ==
\[\begin{tikzcd}
  {\C'} & \C \\
  {\N((\h\C)')} & {\N(\h\C)}
  \arrow[from=1-1, to=1-2]
  \arrow[dashed, from=1-1, to=2-1]
  \arrow[from=1-2, to=2-2]
  \arrow[from=2-1, to=2-2]
  \arrow["\lrcorner"{anchor=center, pos=0.125}, draw=none, from=1-1, to=2-2]
\end{tikzcd}\]

\begin{remark} \label{rem.1.2.11.1}
  部分$\infty$圏ではなく, 部分圏という用語を用いていることに注意. 
  これは任意の部分圏がある圏の脈体で表せるという意味ではない (これは偽である). 
\end{remark}

$\C$を$\infty$圏, $\h\C$を$\C$のホモトピー圏とする. 
$(\h\C)'$が$\h\C$の充満部分圏のとき, $\C'$を$\C$の充満部分圏(full subcategory)という. 
\footnote{
  $\C'$が$\C$の充満部分圏のとき, $\C'$は$\C'$に属する$\C$の対象のなす集合$\C'_0$より定まる. 
  このことから, $\C'$を$\C'_0$で貼られる$\C$の部分圏ということもある. 
}
$\C$を$\infty$圏, $\C'$を$\C$の部分圏とする. 
$\infty$圏の定義より, 包含$C' \hookrightarrow \C$は忠実充満である. 

$\infty$圏の部分圏は内ファイブレーション(2章を参照)を用いて表すことができる.

\begin{lemma} \label{prop.subcategory_equal_inner_fibration}
  $\C$を$\infty$圏, $\C'$を$\C$の単体的部分集合とする. 
  このとき, 次は全て同値である. 
  \begin{enumerate}
    \item $\C'$は$\C$の部分圏である. 
    \item 包含$\C' \hookrightarrow \C$は内ファイブレーションである. 
  \end{enumerate}
  特に, $\C'$は$\infty$圏である. 
\end{lemma}

% \begin{example} \label{eg.subcategory_of_nerve}
%   $\C$を通常の圏, $\C'$を$\C$の通常の部分圏とする. 
%   包含$\N(\C') \hookrightarrow \N(\C)$は内ファイブレーションなので, $\N(\C')$は$\N(\C)$の部分圏である. 
% \end{example}

% \begin{remark} \label{rem.pullback_of_subcategory}
%   $F : \C \to \D$を$\infty$圏の関手, $\D'$を$\D$の部分圏とする. 
%   内ファイブレーションはプルバックで閉じるので, 逆像$F^{-1}(\D')$は$\C$の部分圏である. 
% \end{remark}

% $\N(\C)$の任意の部分圏がこのように表されることを示す. 
% つまり, \cref{def.subcategory}の部分圏は通常の部分圏$の\infty$圏の枠組みにおける一般化と思える.

% \begin{proposition}
%   $\C$を$\infty$圏, $\h\C$を$\C$のホモトピー圏とする. 
%   自然な射$F : \C \to \N(\h\C)$に対して, 構成$(\D \subset \h\C) \mapsto (F^{-1}(\N\D) \subset \C)$は, $\h\C$の通常の部分圏と$\C$の部分圏の間の全単射を定める. 
% \end{proposition}

% \begin{proof}
%   $\D$を$\h\C$の部分圏とする. 
%   \cref{eg.subcategory_of_nerve}より, $\N\D$は$\N(\h\C)$の部分圏である. 
%   \cref{rem.pullback_of_subcategory}より, $F^{-1}(\N\D)$は$\C$の部分圏である.
%   $F : \C \to \N(\h\C)$は単体的集合のepi射なので, $\D$は$F^{-1}(\N\D)$から一意に定まる. 
%   (途中)
% \end{proof}

\end{document}