\RequirePackage{plautopatch}
\documentclass[uplatex, a4paper, 14Q, dvipdfmx]{jsreport}
\usepackage{docmute}
\usepackage{../../mypackage}

\setcounter{secnumdepth}{4}
\title{1.2.4 高次圏における対象, 射, 同値}
\author{Keima Akasaka}
\date{\today}

\begin{document}

% \maketitle

\setcounter{chapter}{1}
\setcounter{section}{2} 
\setcounter{subsection}{3}   
\setcounter{subsubsection}{1}

\subsection{高次圏における対象, 射, 同値}

通常の圏と同様に, 高次圏における対象や射を定義する. 
$\C$が単体的圏か位相的圏のとき, 対象や射はそれぞれの圏における通常の対象や射とすればよい. 
$\C$が$\infty$圏のときは次のように定義する. 

$S$を単体的集合とする. 
$S$の点$\Delta^0 \to S$を$S$の対象(object)という.
$S$の辺$\Delta^1 \to S$を$S$の射(morphism)という.
$S$の対象$X$に対して, $s_0(X) : X \to X$を$X$上の恒等射(identity morphism)といい, $\id_X$と表す.

$\C$を$\infty$圏, $\h\C$を$\C$のホモトピー圏, $f : X \to Y$を$\C$の射とする. 
$f$が$\h\C$における同型射のとき, $f$を同値(equivalence)という.
$\C$の対象$X,Y$が同値で結ばれるとき, $X$と$Y$は同値(equivalent)であるという.

位相的圏$\C$における射$f$が同値であることは, $f$が同型であることよりも次の意味で弱い. 

\begin{proposition} \label{prop.1.2.4.1}
  $\C$を位相的圏, $f : X \to Y$を$\C$の射とする. 
  このとき, 次はすべて同値である.
  \begin{enumerate}
    \item $f$は$\C$における同値である. 
    \item $f$はホモトピー同値$g : Y \to X$を持つ. 
    \item $\C$の任意の対象$W$に対して, 写像$f \circ - : \Map_\C(W,X) \to \Map_\C(W,Y)$はホモトピー同値である. 
    \item $\C$の任意の対象$W$に対して, 写像$f \circ - : \Map_\C(W,X) \to \Map_\C(W,Y)$は弱ホモトピー同値である. 
    \item $\C$の任意の対象$Z$に対して, 写像$- \circ f : \Map_\C(Y,Z) \to \Map_\C(X,Z)$はホモトピー同値である.
    \item $\C$の任意の対象$Z$に対して, 写像$- \circ f : \Map_\C(Y,Z) \to \Map_\C(X,Z)$は弱ホモトピー同値である. 
  \end{enumerate}
\end{proposition}

\begin{proof}
  (2)は(1)の言いかえである. 
  (2)$\Rightarrow$(3)$\Rightarrow$(4)$\Rightarrow$(1)を示す. 
  (2)$\Rightarrow$(5)$\Rightarrow$(6)$\Rightarrow$(1)も同様である. 
  (2)から(3)を示す. 
  $g$を$f$のホモトピー逆射とする. 
  このとき, $g$から定まる写像$- \circ g : \Map_\C(Z,Y) \to \Map_\C(Z,X)$は(3)の$f \circ -$のホモトピー逆射である. 
  (3)から(4)は古典的なホモトピー論から従う. 
  (4)から(1)を示す. 
  (4)を満たすとき, $f \circ - : \Map_\C(W,X) \to \Map_\C(W,Y)$は$\h\C$における同型である. 
  つまり, $\h\C$において$X$と$Y$は同型である. 
  よって, $f$は$\C$における同値である. 
\end{proof}

\begin{example} \label{eg.1.2.4.2}
  $\C$をCW複体の圏とし, 各射集合$\Map_{\C}(X,Y)$にコンパクト開位相によって位相空間を入れることで, $\C$を位相的圏とみなす.
  $\C$の対象$X,Y$が同値であることと, $X,Y$がホモトピー同値であることと同値である.
\end{example}

次の命題は$\infty$圏の枠組みにおける同値を特徴づける定理である. 
証明は2.1.2節で行う. 

\begin{proposition}[Joyal] \label{prop.1.2.4.3}
  $\C$を$\infty$圏, $\phi : \Delta^1 \to \C$を$\C$の射とする. 
  このとき, 次は同値である.
  \begin{enumerate}
    \item $\phi$は同値である. 
    \item 任意の$n \geq 2$と$f_0|_{\Delta^{\{0,1\}}} = \phi$を満たす射$f_0 : \Lambda^n_0 \to \C$に対して, $f_0$から$\Delta^n$への拡張が存在する. 
  \end{enumerate}
\end{proposition}

$\infty$圏における同値は外部角体の拡張条件で表せる. 

\begin{lemma}
  $\C$を$\infty$圏, $f : x \to y$を$\C$の射とする. 
  このとき, 次は同値である. 
  \begin{enumerate}
    \item $f$は同値である. 
    \item 次のように表せる外部角体$\sigma^L_0 : \Lambda^2_0 \to \C$と$\sigma^R_0 : \Lambda^2_2 \to \C$
    % https://q.uiver.app/#q=WzAsNixbMCwxLCJ4Il0sWzIsMSwieCJdLFsxLDAsInkiXSxbMywxLCJ5Il0sWzUsMSwieSJdLFs0LDAsIngiXSxbMCwxLCJcXGlkX3giLDJdLFswLDIsImYiXSxbMyw0LCJcXGlkX3kiLDJdLFs1LDQsImYiXV0=
    \[\begin{tikzcd}
      & y &&& x \\
      x && x & y && y
      \arrow["{\id_x}"', from=2-1, to=2-3]
      \arrow["f", from=2-1, to=1-2]
      \arrow["{\id_y}"', from=2-4, to=2-6]
      \arrow["f", from=1-5, to=2-6]
    \end{tikzcd}\]
    はそれぞれ2単体$\sigma^L : \Delta^2 \to \C$と$\sigma^R : \Delta^2 \to \C$に拡張できる. 
  \end{enumerate}
\end{lemma}

\begin{proof}
  (2)を満たすと仮定する.
  $\sigma^L_0$が$\sigma^L$に拡張できるとき, $f$は$\h\C$において左逆射を持つ. 
  $\sigma^R_0$が$\sigma^R$に拡張できるとき, $f$は$\h\C$において右逆射を持つ.
  よって, $f$は$\C$における同値である. 

  (1)を満たすと仮定する. 
  このとき, ある1単体$g : y \to x$が存在して, $[fg]$と$[gf]$はそれぞれ$\h\C$における恒等射である. 
  つまり, 次のような2単体がそれぞれ存在する. 
  % https://q.uiver.app/#q=WzAsNixbMCwxLCJ4Il0sWzIsMSwieCJdLFsxLDAsInkiXSxbMywxLCJ4Il0sWzUsMSwieCJdLFs0LDAsIngiXSxbMCwxLCJoIiwyXSxbMCwyLCJmIl0sWzMsNCwiXFxpZF94IiwyXSxbNSw0LCJcXGlkX3giXSxbMiwxLCJnIl0sWzMsNSwiaCJdXQ==
  \[\begin{tikzcd}
    & y &&& x \\
    x && x & x && x
    \arrow["h"', from=2-1, to=2-3]
    \arrow["f", from=2-1, to=1-2]
    \arrow["{\id_x}"', from=2-4, to=2-6]
    \arrow["{\id_x}", from=1-5, to=2-6]
    \arrow["g", from=1-2, to=2-3]
    \arrow["h", from=2-4, to=1-5]
  \end{tikzcd}\]
  また, $g$の退化する2単体$s_1(g)$から, 次のように表せる射$\Lambda^3_2 \to \C$が存在する. 
  % https://q.uiver.app/#q=WzAsNCxbMCwyLCJ4Il0sWzIsMiwieCJdLFsyLDEsInkiXSxbMSwwLCJ4Il0sWzAsMSwiaCIsMl0sWzAsMiwiZiIsMl0sWzIsMSwiZyJdLFsxLDMsIlxcaWRfeCIsMCx7ImxhYmVsX3Bvc2l0aW9uIjo2MH1dLFswLDMsIlxcaWRfeCJdLFsyLDMsImciLDJdXQ==
  \[\begin{tikzcd}
    & x \\
    && y \\
    x && x
    \arrow["h"', from=3-1, to=3-3]
    \arrow["f"', from=3-1, to=2-3]
    \arrow["g", from=2-3, to=3-3]
    \arrow["{\id_x}"{pos=0.6}, from=3-3, to=1-2]
    \arrow["{\id_x}", from=3-1, to=1-2]
    \arrow["g"', from=2-3, to=1-2]
  \end{tikzcd}\]
  $\C$は$\infty$圏なので, これは$\Delta^3 \to \C$に拡張できる. 
  このとき, 2単体$\Delta^{\{0,1,3\}}$は$\sigma^L$とみなせる. 
  $\sigma^R$に対しても同様である. 
\end{proof}

\end{document}