\RequirePackage{plautopatch}
\documentclass[uplatex, a4paper, 14Q, dvipdfmx]{jsreport}
\usepackage{docmute}
\usepackage{../../mypackage}

\title{1. 高次圏論の概要}
\author{Keima Akasaka}
\date{\today}

\begin{document}

\maketitle

\chapter{高次圏論の概要}

この章の1つ目の目的は, 高次圏論の一般的な導入をすることである. 
まず, 位相的圏を用いた最も直感的なアプローチから始める. 
この方法は理解しやすいが, 圏論における様々な構成をおこなうときには扱いにくい. 
この問題を解決するために, より適した定式化である$\infty$圏を導入する. 
この定式化は, 圏論のアイデアが適応されやすい, より便利な枠組みである. 
1.1.1節の目標は, 両方のアプローチを紹介し, それらが互いに等価であることを説明することである. 
この等価性の証明は, 2.2節で証明する定理1.1.5.13によっている. 

この章の2つ目の目的は, $\infty$圏の定式化をどのように扱うかを伝えることである. 
1.2節では, 通常の圏論における重要な概念の多くを$\infty$圏の枠組みまで一般化したものを紹介する.
手っ取り早く説明を進めるために, 難しい証明は本書の後の章に先送りにする. 
この章を読んだ後には, 細かい議論に時間を割きたくない読者は, 5章やそれ以降で述べられる基本的なアイデアのいくつかは(少なくとも概要は)理解できるようになっているはずである. 

\end{document}