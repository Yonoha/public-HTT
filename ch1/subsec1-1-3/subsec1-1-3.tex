\RequirePackage{plautopatch}
\documentclass[uplatex, a4paper, 14Q, dvipdfmx]{jsreport}
\usepackage{docmute}
\usepackage{../../mypackage}

\setcounter{secnumdepth}{4}
\title{1.1.3 位相的圏の同値}
\author{Keima Akasaka}
\date{\today}

\begin{document}

% \maketitle

\setcounter{chapter}{1}
\setcounter{section}{1} 
\setcounter{subsection}{2}   
\setcounter{subsubsection}{1}

\subsection{位相的圏の同値}

1.1.1節と1.1.2節で, 高次圏論の基礎付けの方法として位相的圏と単体的集合の2つを見た. 
この2つの定式化が等価であることは後で見る. 
しかし, この等価性は高次圏のレベルで理解されるべきである. 
古典的なホモトピー論と同様に, 扱う対象は位相空間または単体的集合をとることにする. 
「任意のKan複体はある位相空間の特異単体と同型である」という主張や「任意のCW複体はある単体的集合の幾何学的実現と同型である」という主張はともに成立しない. 
しかし, 後者において「同型」を「ホモトピー同値」に置き換えると, 両方の主張は成立する. 
この考えを高次圏論へのアプローチとして用いる. 
よって, 単体的集合の圏において, ホモトピー同値に対応する概念を考える必要がある. 
$F : \C \to \D$を位相的圏の関手としたとき, $F$が同じ高次圏を表すという意味で同値であるために, どのようなことを課せばよいだろうか.

最もナイーブな答えは位相的圏が同型のときに同値とみなすというものである.
しかし, これは通常の圏論における圏同型と同様に, 条件として強すぎる. 
よって, 次に圏同値が考えられる. 

\begin{definition}[強同値]
  $F : \C \to \D$を位相的圏の関手とする.
  $F$が豊穣圏の関手として圏同値のとき, $F$を強同値(strong equivalence)という.
\end{definition}

豊穣圏の強同値は通常の圏に制限したときは通常の圏同値と一致する.
しかし, この定義も条件として強すぎる.
高次圏$\C$の対象$X,Y$に対して, 射空間$\Map_{\C}(X,Y)$はホモトピー同値を除いてwell-definedであるべきだからである. 

\begin{definition}[位相的圏のホモトピー圏] \label{def.1.1.3.2}
  位相的圏$\C$に対して, 通常の圏$\h\C$を次のように定義し, $\C$のホモトピー圏(homotopy category)という.
  \begin{itemize}
    \item $\h\C$の対象は$\C$の対象と同じ.
    \item $\h\C$の任意の対象$X,Y$に対して, $\Hom_{\h\C}(X,Y) := \pi_0 \Hom_\C(X,Y)$.
    \item $\h\C$における射の合成は$\C$における射の合成に関手$\pi_0$を適応させて得られる対応.
  \end{itemize}
\end{definition}

\begin{example}[空間のホモトピー圏] \label{eg.1.1.3.3}
  $\C$をCW複体の位相的圏とする. 
  ここで, $\Map_\C(X,Y)$は$X$から$Y$への連続写像の集合にコンパクト開位相を入れたものとする. 
  このとき, $\C$のホモトピー圏を空間のホモトピー圏(homotopy category of spaces)といい, $\H$と表す. 
\end{example}

空間のホモトピー圏$\H$は次のように定義することもできる. 
次の定義は高次圏論において非常に重要である. 
まず, 古典的なホモトピー論で重要な概念を復習する. 

\begin{definition} \label{def.1.1.3.4}
  $f : X \to Y$を位相空間の連続写像とする. 
  $\pi_0f : \pi_0X \to \pi_0Y$が全単射かつ, $X$の任意の点$x$と$i \geq 1$に対して, ホモトピー群の写像
  \begin{align*}
    \pi_i(X,x) \to \pi_i(Y,f(x))
  \end{align*}
  が同型のとき, $f$を弱ホモトピー同値(weak homotopy equivalence)という. 
\end{definition}

$\CGWH$の任意の対象$X$に対して, あるCW複体$X'$と弱ホモトピー同値$\phi : X' \to X$が存在する. 
この$X'$は一意ではないが, ホモトピー同値を除いて一意に定まる. 
よって, 構成$X \mapsto X'$は関手$\theta : \CGWH \to \H$を定める. 
$\theta$の定義から, $\theta$は$\CGWH$における弱ホモトピー同値を$\H$における同型射にうつす. 
CW複体の間の弱ホモトピー同値はホモトピー同値を持つので, $\H$は$\CGWH$にすべての弱ホモトピー同値を添加した圏とみなせる. 

関手$\theta : \CGWH \to \H$は有限直積を保つ. 
注意.A.1.4.3より, $\H$豊穣圏が得られる. 
$\H$が$\CGWH$にすべての弱ホモトピー同値を添加した圏とみなせることから, この$\H$豊穣圏を位相的圏$\C$のホモトピー圏(homotopy category)といい, $\h\C$と表す. 

この意味のホモトピー圏$\h\C$は次のように具体的に表せる. 
\begin{itemize}
  \item $\h\C$の対象は$\C$と同じ.
  \item $\h\C$の任意の対象$X,Y$に対して, $\Map_{\h\C}(X,Y) := [\Map_\C(X,Y)]$.
  \item $\h\C$における射の合成は$\C$における射の合成に関手$\theta : \CGWH \to \H$を適応させて得られる対応.
\end{itemize}

\begin{remark} \label{rem.1.1.3.5}
  位相的圏$\C$に対して, ホモトピー圏$\h\C$を2種類の方法で構成した. 
  1つは通常の圏として, もう1つは$\H$豊穣圏としてである. 
  これらが等価であることは, 任意の位相空間$X$に対して自然な全単射$\pi_0X \cong \Map_\H(\ast,[X])$が存在することから従う.
\end{remark}

位相的圏$\C$のホモトピー圏$\h\C$は$\C$の位相的な射空間の情報は忘れて, その(弱)ホモトピー型のみを抽出したような圏とみなせる. 
本質的に重要なのはホモトピー型の情報であり, 位相的圏の同値はこのレベルで考えるべきであることが分かる. 

\begin{definition}[位相的圏の同値] \label{def.1.1.3.6}
  $F : \C \to \D$を位相的圏の関手とする. 
  誘導される関手$\h F : \h\C \to \h\D$が$\H$豊穣圏として圏同値のとき, $F$を同値(equivalence)という. 
\end{definition}

\begin{remark} \label{rem.1.1.3.7}
  $F$が同値であることと, 次の2つを満たすことは同値である.
  \begin{itemize}
    \item $\C$の任意の対象$X,Y$に対して, $\Map_\C(X,Y) \to \Map_\D(FX,FY)$は弱ホモトピー同値である. 
    \item $\D$の任意の対象$Y$に対して, $\C$のある対象$X$が存在して, $\h\D$において$FX$と$Y$は同型である. 
  \end{itemize}
\end{remark}

% 2つの位相的圏の間に同値が存在するとき, これらは同値(equivalent)であるという. 
定義から, 位相的圏の関手$F : \C \to \D$が同値であることと, $\h F : \h\C \to \h\D$が圏同値であることは同値である. 
つまり, ホモトピー圏$\h\C$は$\C$の不変量である. 
しかし, $\h\C$は$\C$によって同値の違いを除いても一意に定まるわけではない. 

\end{document}