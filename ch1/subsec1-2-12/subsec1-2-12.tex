\RequirePackage{plautopatch}
\documentclass[uplatex, a4paper, 14Q, dvipdfmx]{jsreport}
\usepackage{docmute}
\usepackage{../../mypackage}

\setcounter{secnumdepth}{4}
\title{1.2.12 始対象と終対象}
\author{Keima Akasaka}
\date{\today}

\begin{document}

% \maketitle

\setcounter{chapter}{1}
\setcounter{section}{2} 
\setcounter{subsection}{11}   
\setcounter{subsubsection}{1}

\subsection{始対象と終対象}

通常の圏論と同様に, 高次圏における終対象と始対象を定義する. 
位相的圏$\C$に対して, 射空間の位相を無視して$\C$を通常の圏とみなしたときの終対象を$\C$における終対象と定義することが考えられる. 
しかし, この定義は強すぎることが分かる. 
例えば, $\CGWH$において, 1点からなる位相空間$\ast$はこの意味の終対象である.
しかし, $\ast$と同値な(つまり, 任意の可縮空間)位相空間は$\ast$と同相ではなく, $\CGWH$における終対象ではない. 
$\infty$圏における概念は同値で保たれるべきであるので, これよりも弱い定義が必要である.  

\begin{definition}[終対象] \label{def.1.2.12.1}
  $\C$を単体的集合(位相的圏, 単体的圏), $\C$のホモトピー圏$\h\C$を$\H$豊穣圏とみなす. 
  $\C$の対象$X$が$\h\C$における通常の終対象のとき, $X$を終対象(final object)という. 
\end{definition}

\begin{remark} \label{rem.1.2.12.2}
  \Cref{def.1.2.12.1}の定義において, ホモトピー圏の情報しか用いていないので, ここまでで紹介した操作と同値に対して, 終対象性は不変である. 
\end{remark}

$\infty$圏の枠組みにおいては, より良い定義として強終対象がある. 
\Cref{cor.1.2.12.5}で, $\infty$圏において, これが終対象と同値な定義であることを見る. 

\begin{definition}[強終対象] \label{def.1.2.12.3}
  $\C$を単体的集合, $X$を$\C$の対象とする. 
  射影$\C_{/ X} \to \C$が自明なKanファイブレーションのとき, $X$を強終対象(strongly final object)という. 
\end{definition}

% \begin{lemma} \label{prop.terminal_equal_extension_property}
%   $\C$を単体的集合, $X$を$\C$の対象とする. 
%   このとき, 次は同値である. 
%   \begin{enumerate}
%     \item $X$は強終対象である.
%     \item 任意の$n \geq 1$に対して, 次の図式はリフトを持つ.
%     % https://q.uiver.app/#q=WzAsNCxbMSwwLCJcXHBhcnRpYWwgXFxEZWx0YV5uIl0sWzEsMSwiXFxEZWx0YV5uIl0sWzIsMCwiXFxDIl0sWzAsMCwiXFxEZWx0YV57XFx7blxcfX0iXSxbMCwxLCIiLDAseyJzdHlsZSI6eyJ0YWlsIjp7Im5hbWUiOiJob29rIiwic2lkZSI6InRvcCJ9fX1dLFswLDJdLFsxLDIsIiIsMCx7InN0eWxlIjp7ImJvZHkiOnsibmFtZSI6ImRhc2hlZCJ9fX1dLFszLDBdLFszLDIsIlgiLDAseyJjdXJ2ZSI6LTJ9XV0=
%     \[\begin{tikzcd}
%       {\Delta^{\{n\}}} & {\partial \Delta^n} & \C \\
%       & {\Delta^n}
%       \arrow[hook, from=1-2, to=2-2]
%       \arrow[from=1-2, to=1-3]
%       \arrow[dashed, from=2-2, to=1-3]
%       \arrow[from=1-1, to=1-2]
%       \arrow["X", curve={height=-12pt}, from=1-1, to=1-3]
%     \end{tikzcd}\] 
%   \end{enumerate}
% \end{lemma}

% \begin{proof}
%   $\C_{/ X}$の定義から, (2)の図式のリフト性は次の図式のリフト性と同値である. 
%   % https://q.uiver.app/#q=WzAsNCxbMCwwLCJcXHBhcnRpYWwgXFxEZWx0YV57bi0xfSJdLFswLDEsIlxcRGVsdGFee24tMX0iXSxbMSwwLCJcXENfey8gWH0iXSxbMSwxLCJcXEMiXSxbMCwxLCIiLDAseyJzdHlsZSI6eyJ0YWlsIjp7Im5hbWUiOiJob29rIiwic2lkZSI6InRvcCJ9fX1dLFswLDJdLFsxLDIsIiIsMCx7InN0eWxlIjp7ImJvZHkiOnsibmFtZSI6ImRhc2hlZCJ9fX1dLFsyLDNdLFsxLDNdXQ==
%   \[\begin{tikzcd}
%     {\partial \Delta^{n-1}} & {\C_{/ X}} \\
%     {\Delta^{n-1}} & \C
%     \arrow[hook, from=1-1, to=2-1]
%     \arrow[from=1-1, to=1-2]
%     \arrow[dashed, from=2-1, to=1-2]
%     \arrow[from=1-2, to=2-2]
%     \arrow[from=2-1, to=2-2]
%   \end{tikzcd}\]
% \end{proof}

\begin{proposition}  \label{prop.1.2.12.4}
  $\C$を$\infty$圏, $Y$を$\C$の対象とする. 
  このとき, 次は同値である. 
  \begin{enumerate}
    \item $Y$は強終対象である. 
    \item $\C$の任意の対象$X$に対して, $\Hom^\R_\C(X,Y)$は可縮なKan複体である.
  \end{enumerate}
\end{proposition}

\begin{proof} 
  $\Hom^\R_\C(X,Y)$の定義より, $\Hom^\R_\C(X,Y)$はファイバー$(\C_{/ Y})_X = \C_{/ Y} \times_\C \{X\}$と同一視できる. 

  (1)から(2)を示す. 
  $Y$が強終対象のとき, 射影$p : \C_{/ Y} \to \C$は自明なKanファイブレーションである. 
  自明なKanファイブレーションの集まりはプルバックで閉じるので, ファイバー$\C_{/ Y} \times_\C \{X\}$は可縮なKan複体である. 

  (2)から(1)を示す. 
  $\C$の任意の対象$X$に対して, $\Hom^\R_\C(X,Y) = (\C_{/ Y})_X$が可縮であるとする. 
  命題2.1.2.1より, $p$は右ファイブレーションである. 
  補題2.1.3.4より, $p$は自明なKanファイブレーションである. 
\end{proof}

\begin{corollary} \label{cor.1.2.12.5}
  $\C$を単体的集合とする. 
  $\C$における任意の強終対象は終対象である. 
  逆は$\C$が$\infty$圏のときに成立する.
\end{corollary}

\begin{remark} \label{rem.1.2.12.6}
  終対象の双対として, $\infty$圏における始対象が考えられる.
\end{remark}

\begin{example} \label{eg.1.2.12.7}
  $\C$を通常の圏とする. 
  $\N(\C)$における対象が終(始)対象であることと, $\C$において通常の意味で終(始)対象であることは同値である. 
  これは圏同値$\h\N(\C) \cong \C$が成立することから従う. 
\end{example}

\begin{remark} \label{rem.1.2.12.8}
  \Cref{def.1.2.12.3}は$\C$が$\infty$圏の場合でないと意味をなさない. 
  例えば, $\C$が$\infty$圏でないとき, $\C$の強終点の集まりが同値で安定とは限らない.
\end{remark}

通常の圏における終対象は同型を除いて一意に定まる. 
$\infty$圏における終対象も同様の主張ができるが, 「一意に」という概念をホモトピー論的な言葉に置き換える必要がある. 
実際, 終対象は可縮な空間の選択を除いて一意に定まる. 

\begin{proposition}[Joyal] \label{prop.1.2.12.9}
  $\C$を$\infty$圏, $\C'$を$\C$の終対象のなす$\C$の充満部分圏とする. 
  このとき, $\C'$は空または可縮なKan複体である.  
\end{proposition}

\begin{proof}
  $\C'$が空でないとする. 
  次の図式がリフトを持つことを示せばよい. 
  % https://q.uiver.app/#q=WzAsMyxbMCwwLCJcXHBhcnRpYWwgXFxEZWx0YV5uIl0sWzAsMSwiXFxEZWx0YV5uIl0sWzEsMCwiXFxDJyJdLFswLDEsIiIsMCx7InN0eWxlIjp7InRhaWwiOnsibmFtZSI6Imhvb2siLCJzaWRlIjoidG9wIn19fV0sWzAsMl0sWzEsMiwiIiwwLHsic3R5bGUiOnsiYm9keSI6eyJuYW1lIjoiZGFzaGVkIn19fV1d
  \[\begin{tikzcd}
    {\partial \Delta^n} & {\C'} \\
    {\Delta^n}
    \arrow[hook, from=1-1, to=2-1]
    \arrow[from=1-1, to=1-2]
    \arrow[dashed, from=2-1, to=1-2]
  \end{tikzcd}\]
  $n=0$のとき, $\C'$は空でない仮定から従う. 
  $n \geq 1$のとき, $\partial \Delta^n$の対象$\Delta^{\{n\}}$が終対象にうつることから従う. 
\end{proof}

\end{document}