\RequirePackage{plautopatch}
\documentclass[uplatex, a4paper, 14Q, dvipdfmx]{jsreport}
\usepackage{docmute}
\usepackage{../../mypackage}

\setcounter{secnumdepth}{4}
\title{1.2.16 空間の\texorpdfstring{$\infty$}{infty}圏}
\author{Keima Akasaka}
\date{\today}

\begin{document}

% \maketitle

\setcounter{chapter}{1}
\setcounter{section}{2} 
\setcounter{subsection}{15}   
\setcounter{subsubsection}{1}

\subsection{空間の\texorpdfstring{$\infty$}{infty}圏}

通常の圏論において, 多くの圏は$\Set$で豊穣された圏であった. 
高次圏論におけるこのアナロジーは空間で豊穣された$\infty$圏である.

\begin{definition}[空間の$\infty$圏] \label{def.1.2.16.1}
  小Kan複体のなす$\sSet$の充満部分圏を$\Kan$と表す. 
  $\Kan$を単体的圏とみなし, $\Kan$の単体的脈体$\mathfrak{N}(\Kan)$を空間の$\infty$圏($\infty$-category of spaces)といい, $\S$と表す. 
\end{definition}

\begin{remark} \label{rem.1.2.16.2}
  $\Kan$の任意の対象$X,Y$に対して, 単体的集合$\Map_{\Kan}(X,Y) = Y^X$はKan複体である. 
  命題1.1.5.10より, $\S$は$\infty$圏である. 
\end{remark}

\begin{remark} \label{rem.1.2.16.3}
  空間の$\infty$圏として, CW複体のなす圏の位相的脈体なども考えられる. 
  このようなものは全て$\S$と等価であることが分かる. 
  \Cref{def.1.2.16.1}の定義は$\infty$圏におけるYonedaの補題を示すときに扱いやすいからである. 
  詳しくは5.1.3節で議論する.
\end{remark}

\begin{remark} \label{rem.1.2.16.4}
  $\S$は小Kan複体のなす圏に対して定義されていた. 
  小とは限らないすべてのKan複体に対して定義される空間の$\infty$圏を$\hat{\S}$と表す. 
  $\S$は大きい$\infty$圏であるが, $\hat{\S}$はより大きな$\infty$圏であることを後で見る. 
\end{remark}

\end{document}