\RequirePackage{plautopatch}
\documentclass[uplatex, a4paper, 14Q, dvipdfmx]{jsreport}
\usepackage{docmute}
\usepackage{../../mypackage}

\setcounter{secnumdepth}{4}
\title{1.2.2 高次圏における射空間}
\author{Keima Akasaka}
\date{\today}

\begin{document}

% \maketitle

\setcounter{chapter}{1}
\setcounter{section}{2} 
\setcounter{subsection}{1}   
\setcounter{subsubsection}{1}

\subsection{高次圏における射空間}

通常の圏$\C$の任意の対象$X,Y$に対して, 射集合$\Hom_\C(X,Y)$が定義されている. 
高次圏$\C$においても同様に, 射空間$\Map_\C(X,Y)$が定義される. 
位相的圏や単体的圏においては, $\Map_\C(X,Y)$は豊穣圏の枠組みとして定義されている.
しかし, $\infty$圏における$\Map_\C(X,Y)$の定義は少し非自明である. 
この節の目標は, $\infty$圏における射空間の定義を理解することである. 
$\infty$圏における射空間はホモトピー圏のレベルで定義すれば十分であることが分かる. 

\begin{definition}[単体的集合の射空間] \label{def.1.2.2.1}
  $S$を単体的集合, $x,y$を$S$の任意の点とする. 
  $S$のホモトピー圏$\h S$を$\H$豊穣圏とみなす. 
  このとき, $\Map_S(x,y) := \Map_{\h S}(x,y)$を$S$における$x$から$y$への射の空間を表す$\H$の対象とする. 
\end{definition}

\begin{remark} \label{rem.1.2.2.2}
  $S$を単体的集合とする. 
  $X,Y$が$S$の点のとき, $\Map_S(X,Y)$は\cref{def.1.2.2.1}の意味で, $\H$の対象である. 
  一方, $X,Y$が$(\sSet)_{/S}$の対象
  \footnote{
    $ (\sSet)_{\S}$の対象は単体的集合の射$X \to S$であるが, このような省略を用いる. 
  }
  のときは
  \begin{align*}
    Y^X \times_{S^X} \{X \to S\}
  \end{align*}
  を$\Map_S(X,Y)$と表す.
\end{remark}

単体的集合$S$とその点$x,y$に対して, どのように$\Map_S(x,y)$を計算すればいいのだろうか. 
$\Map_S(x,y)$は$\H$の対象として定義されたが, $\Map_S(x,y)$を表すような単体的集合$M$を選ぶ必要がある.
このような$M$として$\Map_{\mathfrak{C}[S]}(x,y)$がまず考えられる.
この定義の利点は$S$が$\infty$圏でないときも計算することができ, 強結合的な結合則を備えていることである.
しかし, $\Map_{\mathfrak{C}[S]}(x,y)$の構成は複雑であり, 一般にはKan複体にはならない. 
そのため, ホモトピー群のような代数的な不変量を取り出すことも難しい. 

この欠点に対処するために,  $\Map_S(x,y)$のホモトピー型を表すような単体的集合$\Hom^\R_S(x,y)$を定義する. 
これは$S$が$\infty$圏の時のみに定義される. 
% \footnote{
%   $\Hom^\R_S(x,y)$は一般の単体的集合$S$に対して定義することはできるが, $S$が$\infty$圏のときによい性質を持つことが分かる. 
%   \cref{prop.1.2.2.3}や\cref{rem.1.2.2.4}などを参照. 
% }

$S$を$\infty$圏, $x,y$を$S$の任意の点とする. 
このとき, 単体的集合$\Hom^\R_S(x,y)$を次のように定義し, $x$から$y$への右射空間(space of right morphisms)という. 
\begin{itemize}
  \item 任意の$n \geq 0$に対して, $\Hom^\R_S(x,y)_n$は, $z|_{\Delta^{\{n+1\}}} = y$かつ$z|_{\Delta^{\{0,\cdots,n\}}}$が点$x$の定値単体であるような$S$の$(n+1)$単体$z : \Delta^{n+1} \to S$の集合.
  \item $\Hom^\R_S(x,y)_n$の退化写像や面写像は$S_{n+1}$の退化写像や面写像.
\end{itemize}
$S$が$\infty$圏のとき, 単体的集合$\Hom^\R_S(x,y)$は空間のようにふるまう. 

\begin{proposition} \label{prop.1.2.2.3}
  $\C$を$\infty$圏, $x,y$を$\C$の点とする. 
  このとき, $\Hom^\R_\C(x,y)$はKan複体である.
\end{proposition}

\begin{proof}
  $\Hom^\R_\C(x,y)$の定義より, 任意の$n \geq 2$と$0 < i \leq n$に対して, 次の図式はリフトを持つ. 
  % https://q.uiver.app/#q=WzAsMyxbMCwwLCJcXExhbWJkYV5uX2kiXSxbMSwwLCJcXEhvbV5cXFJfXFxDKHgseSkiXSxbMCwxLCJcXERlbHRhXm4iXSxbMCwxXSxbMCwyLCIiLDIseyJzdHlsZSI6eyJ0YWlsIjp7Im5hbWUiOiJob29rIiwic2lkZSI6InRvcCJ9fX1dLFsyLDEsIiIsMix7InN0eWxlIjp7ImJvZHkiOnsibmFtZSI6ImRhc2hlZCJ9fX1dXQ==
  \[\begin{tikzcd}
    {\Lambda^n_i} & {\Hom^\R_\C(x,y)} \\
    {\Delta^n}
    \arrow[from=1-1, to=1-2]
    \arrow[hook, from=1-1, to=2-1]
    \arrow[dashed, from=2-1, to=1-2]
  \end{tikzcd}\]
  命題1.2.5.1より, $\Hom^\R_\C(x,y)$はKan複体である.
\end{proof}

\begin{remark} \label{rem.1.2.2.4}
  $S$を単体的集合, $x,y,z$を$S$の任意の点とする. 
  一般には, 次のような合成は存在しない. 
  \begin{align*}
    \Hom^\R_S(x,y) \times \Hom^\R_S(y,z) \to \Hom^\R_S(x,z)
  \end{align*}
  しかし, $S$が$\infty$圏のときはこのような合成が定まり, 可縮な空間の選択を除いてwell-definedであることを後で示す. 
  この合成の自然な選択がないことは$\Map_{\mathfrak{C}[S]}(x,y)$に比べて$\Map^\R_S(x,y)$の欠点である. 
  2.2節の目標は, ホモトピー圏$\H$において$\Map^\R_S(x,y)$と$\Map_{\mathfrak{C}[S]}(x,y)$の間に自然な同型が存在することを示すことである. 
  特に, $S$が$\infty$圏のとき, $\Hom^\R_S(x,y)$は$\Map_S(x,y)$を表すことを示す. 
\end{remark}

\begin{remark} \label{rem.1.2.2.5}
  $\Hom^\R_S(x,y)$の定義は自己双対的ではない. 
  $S$を単体的集合, $x,y,z$を$S$の任意の点とする. 
  このとき, 単体的集合$\Hom^\L_S(x,y)$を次のように定義し, $x$から$y$への左射空間(space of left morphisms)という. 
  \begin{align*}
    \Hom^\L_S(x,y) := \Hom^\R_{S^\myop}(y,x)^\myop
  \end{align*}
  つまり, 任意の$n \geq 0$に対して, $\Hom^\L_S(x,y)_n$は$z|_{\Delta^{0}} = x$かつ$z|_{\Delta^{\{1,\cdots,n+1\}}}$が点$y$の定値単体であるような$S$の$(n+1)$単体$z : \Delta^{n+1} \to S$の集合である. 
\end{remark}

一般に, $\Hom^\L_S(x,y)$と$\Hom^\R_S(x,y)$は単体的集合として同型ではない.
しかし, $S$が$\infty$圏のとき, これらはホモトピー同値である. 
ここで, 
\begin{align*}
  \Hom_S(x,y) := \{x\} \times_{S} S^{\Delta^1} \times_S \{y\}
\end{align*}
と定義すると, この定義は自己双対的である. 
このとき, 次の自然な包含が存在する. 
\begin{align*}
  \Hom^\R_S(x,y) \hookrightarrow \Hom_S(x,y) \hookleftarrow \Hom^\L_S(x,y)
\end{align*}
系.4.2.1.8で, $S$が$\infty$圏のとき, これらの包含がホモトピー同値であることを示す. 

\end{document}